\chapter{Statements}
The crux of the SLANG4.net can be summed up in two sentences
\begin{verbatim}
1. Expression is what you evaluate for it's value.
2. Statement is what you execute for it's effect (on variables).
\end{verbatim}
The above two maxims can be converted into a computational structure as follows:
\lstset{style=csharp}
\begin{lstlisting}
// Expression is what you evaluate for it's value
public abstract class Exp
{
	public abstract double Evaluate(RUNTIME_CONTEXT cont);
}

// Statement is what you Execute for it's Effect
// on variables or lack of it
public abstract class Stmt
{
	public abstract bool Execute(RUNTIME_CONTEXT con);
}
\end{lstlisting}

\section{Print statement}
Let us implement a Print statement for the SLANG4.net compiler. The basic idea is as follows you add a class to model a statement and since the class has to inherit from the Stmt ( abstract class ), it ought to implement Execute Method.
\lstset{style=csharp}
\begin{lstlisting}
// Implementation of Print Statement
public class PrintStatement : Stmt
{
	// At this point of time, Print will
	// spit the value of an Expression on the screen.
	private Exp _ex;
	
	// Ctor just stores the expression passed as parameter
	public PrintStatement(Exp ex)
	{
		_ex = ex;
	}

	// Execute method Evaluates the expression and
	// spits the value to the console using
	// Console.Write statement.
	public override bool Execute(RUNTIME_CONTEXT con)
	{
		double a = _ex.Evaluate(con);
		Console.Write(a.ToString());
		return true;
	}
}
\end{lstlisting}
\section{PrintLine statement}
Let us add a PrintLine statement as well. PrintLine implementation is not different from Print statement. The only difference is it emits a new line after the expression value.
\lstset{style=csharp}
\begin{lstlisting}
// Implementation of PrintLine Statement
public class PrintLineStatement : Stmt
{
	private Exp _ex;
	public PrintLineStatement(Exp ex)
	{
		_ex = ex;
	}
	
	// Here we are calling Console.WriteLine to emit
	// an additional new line .
	public override bool Execute(RUNTIME_CONTEXT con)
	{
		double a = _ex.Evaluate(con);
		Console.WriteLine(a.ToString());
		return true;
	}
}
\end{lstlisting}

\section{Changes in the Parser}
Once we have created to classes to implement Print and PrintLine statement, we need to modify our
parser (frontend ) to support the statement in the language.\\
\subsection{Changes in the Lexer}
We are going to add few more tokens to support the Statements in the SLANG.\\
\lstset{style=csharp}
\begin{lstlisting}
public enum TOKEN
{
	ILLEGAL_TOKEN = -1, // Not a Token
	TOK_PLUS = 1, // '+'
	TOK_MUL, // '*'
	TOK_DIV, // '/'
	TOK_SUB, // '-'
	TOK_OPAREN, // '('
	TOK_CPAREN, // ')'
	TOK_DOUBLE, // 'number'
	TOK_NULL, // End of string
	TOK_PRINT, // Print Statement
	TOK_PRINTLN, // PrintLine
	TOK_UNQUOTED_STRING,
	TOK_SEMI // ;
}
\end{lstlisting}
In the Lexer.cs module, we add a new data structure to be used for Keyword lookup.
\lstset{style=csharp}
\begin{lstlisting}
// Keyword Table Entry
public struct ValueTable
{
	public TOKEN tok; // Token id
	public String Value; // Token string
	public ValueTable(TOKEN tok, String Value)
	{
		this.tok = tok;
		this.Value = Value;
	}
}
\end{lstlisting}
In the ctor of Lexer.cs, we will populate an array of ValueTables with Token and it's textual representation as given below.
\lstset{style=csharp}
\begin{lstlisting}
// Keyword Table Entry
public struct ValueTable
{
	_val = new ValueTable[2];
	_val[0] = 
	new ValueTable(TOKEN.TOK_PRINT, "PRINT");
	_val[1] = 
	new ValueTable(TOKEN.TOK_PRINTLN, "PRINTLINE");
}
\end{lstlisting}
\subsection{Grammar}
We need to add a new entrypoint into the RDParser.cs class to support statements. The grammar for the SLANG at this point of time ( to support statement ) is as given below.
\lstset{style=csharp}
\begin{lstlisting}
<stmtlist> := { <statement> }+
<statement> := <printstmt> | <printlinestmt>
<printstmt> := print <expr >;
<printlinestmt>:= printline <expr>;
<Expr> ::= <Term> | Term { + | - } <Expr>
<Term> ::= <Factor> | <Factor> {*|/} <Term>
<Factor>::= <number> | ( <expr> ) | {+|-} <factor>
\end{lstlisting}
\subsection{Parser entry point}
The new entry point to the parser module is as follows.
\lstset{style=csharp}
\begin{lstlisting}
// The new Parser entry point
public ArrayList Parse()
{
	GetNext(); // Get the Next Token
	// Parse all the statements
	return StatementList();
}
\end{lstlisting}
\subsection{Statement List}
The StatementList method implements the grammar given above. The BNF to source code translation is very easy and without much explanation it is given below.
\lstset{style=csharp}
\begin{lstlisting}
private ArrayList StatementList()
{
	ArrayList arr = new ArrayList();
	while (Current_Token != TOKEN.TOK_NULL)
	{
		Stmt temp = Statement();
		if (temp != null)
		{
			arr.Add(temp);
		}
	}
	return arr;
}
\end{lstlisting}
\subsection{Statement}
The method Statement queries the statement type and parses the rest of the statement.
 \lstset{style=csharp}
 \begin{lstlisting}
// This Routine Queries Statement Type
// to take the appropriate Branch...
// Currently, only Print and PrintLine statement
// are supported..
// if a line does not start with Print or PrintLine ..
// an exception is thrown
private Stmt Statement()
{
	Stmt retval = null;
	switch (Current_Token)
	{
		case TOKEN.TOK_PRINT:
			retval = ParsePrintStatement();
			GetNext();
			break;

		case TOKEN.TOK_PRINTLN:
			retval = ParsePrintLNStatement();
			GetNext();
			break;

		default:
			throw new Exception("Invalid statement");
			break;
	}
	return retval;
}

// Parse the Print Staement .. The grammar is
// PRINT <expr> ;
// Once you are in this subroutine, we are expecting
// a valid expression ( which will be compiled ) and a
// semicolon to terminate the line..
// Once Parse Process is successful, 
// we create a PrintStatement
// Object..
private Stmt ParsePrintStatement()
{
	GetNext();
	Exp a = Expr();
	if (Current_Token != TOKEN.TOK_SEMI) {
		throw 
		new Exception("; is expected");
	}
	return new PrintStatement(a);
}
// Parse the PrintLine Staement .. The grammar is
// PRINTLINE <expr> ;
// Once you are in this subroutine, we are expecting
// a valid expression ( which will be compiled ) and a
// semi collon to terminate the line..
// Once Parse Process is successful, we create 
// a PrintLineStatement Object..
private Stmt ParsePrintLNStatement()
{
	GetNext();
	Exp a = Expr();
	if (Current_Token != TOKEN.TOK_SEMI) {
		throw 
		new Exception("; is expected");
	}
	return new PrintLineStatement(a);
}
\end{lstlisting}
\section{Main}
Finally in the callSlang Project, I have invoked these routines to demonstrate how everything is put together.
\lstset{style=csharp}
\begin{lstlisting}
using System;
using System.Collections.Generic;
using System.Linq;
using System.Text;
using System.Collections;
using SLANG_DOT_NET;
namespace CallSLANG
{
	class Program
	{
		static void TestFirstScript()
		{
			string a = 
			"PRINTLINE 2*10;" + "\r\n" 
			+ "PRINTLINE 10;"+"\r\n" 
			+"PRINT 2*10;"+"\r\n";

			RDParser p = 
			new RDParser(a);

			ArrayList arr = 
			p.Parse();
			foreach (object obj in arr)
			{
				Stmt s 
				= obj as Stmt;
				s.Execute(null);
			}
		}
		
		static void TestSecondScript()
		{
			string a = 
			"PRINTLINE -2*10;" + "\r\n" 
			+ "PRINTLINE -10*-1;\r\n PRINT 2*10;\r\n";
			RDParser p = new RDParser(a);
			ArrayList arr = p.Parse();
			foreach (object obj in arr)
			{
				Stmt s = obj as Stmt;
				s.Execute(null);
			}
		}
		
		static void Main(string[] args)
		{
			// TestFirstScript();
			TestSecondScript();
			Console.Read();
		}

	} 
}
\end{lstlisting}